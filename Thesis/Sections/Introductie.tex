\chapter{Introductie}\label{introductie}

Vandaag de dag beschikken we over een enorme hoeveelheid aan digitale informatie. En iedere dag wordt deze hoeveelheid groter en groter.
In deze \textit{``Age of Big Data''} (\cite{lohr2012age}) bestaat de uitdaging erin om uit deze grote hoeveelheid data, door middel van analyse bepaalde inzichten te krijgen.
\\
Vele hebben dit probleem proberen aan te pakken, waarbij men zich bezig hield met structuur brengen in deze grote dataset en zich voornamelijk concentreerde op onderwerp-gebaseerde classificatie. Echter met de opkomst van sociale media, blogs, reviewsites is een groeiende interesse ontstaan voor gevoelsanalyse. Het onderzoek dat hiernaar gebeurd, heeft ook de aandacht van bedrijven en spelen hier ook een rol in. Bijna al het onderzoek dat gebeurd is de afgelopen jaren zijn uitgevoerd op Engelse teksten en er is zeer weinig te vinden over onderzoek naar gevoelsanalyse op het Nederlands. Deels omdat het Engels een wereldtaal is en het Nederlands niet, maar ook door de bedrijven die dergelijk onderzoek binnenshuis houden.
\\
Voor deze bachelorproef kijken we of de bevindingen over gevoelsanalyse op het Engels gelijk toepasbaar zijn op het Nederlands en of er verschillen zijn tussen het Nederlands en het Engels, waarbij men met een gevoelsanalyse rekening mee moet houden.

We hebben op basis van de voorbereiding op de bachelorproef in het 1ste semester en  een verdere verdieping in de literatuur over gevoelsanalyse een selectie gemaakt van technieken die we gaan vergelijken tijdens het experiment.In hoofdstuk \ref{Lectuur} worden deze technieken met hun theoretische achtergrond en de selectie van deze technieken besproken. Vervolgens gaan we in hoofdstuk \ref{Experiment} over naar drie experimenten. We vergelijken in het eerste experiment de resultaten van de technieken op Nederlandse en Engelse filmrecensies. Vervolgens concentreren we ons op de woordenschat en  onderzoeken we de classificatie op basis van een Engels en Nederlands geannoteerde woordenlijst van gevoelens. Als laatste vergelijken we mogelijke invloeden van het onderwerp van een tekst en vergelijken we de prestatie van Nederlandse gevoelsanalyse op verschillende recensies met verschillende onderwerpen. Na het experiment vormen we een conclusie of de gevonden technieken voor gevoelsanalyse op het Engels al dan niet gelijk toepasbaar zijn op het Nederlands. 