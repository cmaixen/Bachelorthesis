\chapter{Conclusie}\label{Conclusie}

In deze voorbereiding hebben we omschreven wat Machine Learning juist omvat. Namelijk het onderzoeken en ontwikkeling van zelflerende algoritmes, die hoofdzakelijk uit drie stappen bestaan, namelijk data verzamelen, verwerken en analyseren. Naargelang het soort data en wat deze weergeeft bestaan er binnen het domein van Machine Learning verschillende technieken met hun specifieke eigenschappen en voorbeelden. Voor situaties waarin we op voorhand over voldoende data beschikken en we duidelijk weten wat deze data betekend, bespraken we in sectie 2.2 verschillende supervised learning technieken die in staat zijn om een hypothese te formuleren op basis van de gegeven data. We maakte een duidelijk onderscheid welke problemen er zich kunnen voordoen zoals een classificatie probleem versus een regressie probleem en hoe men deze moet oplossen.\\ We hebben gezien dat een classificatie probleem zich onderscheidt van een regressie probleem door dat de output van de hypothese zich beperkt tot een kleine set van mogelijkheden, wat bij een regressie probleem een hele reeks van mogelijkheden is. 
Vervolgens hebben we een specifieke techniek besproken, namelijk de vector space methode. Een methode die van toepassing is bij text mining. Deze methode kan men op verschillende manieren verfijnen. Zo raadde we aan in sectie 3.1 om document pre-processing toe te passen op de dataset voor de verwerking van de data. Verder werd er ook aangeraden in sectie 3.2.1 om de standaard vector space methode te optimaliseren met technieken zoals Latent Semantic Analysis (LSA) en Term weighting. Ten slotte hebben we een proefopstelling opgesteld waarbij de techniek LSA wordt toepast en de effici\"entie en werking van Latent Semantic Analysis nogmaals wordt bevestigd.

