\chapter{Conclusie}\label{Conclusie}

In deze bachelorproef hadden we als onderwerp gevoelsanalyse in het Nederlands. Hierbij zagen zowel in hoofdstuk \ref{Lectuur} de theoretische kant als in hoofdstuk \ref{Experiment} de praktische kant. De theorie gaf ons de basis voor het onderzoek, waarbij werd uitgelegd hoe we de gevoelsanalyse konden uitvoeren aan de hand van machine learning. In sectie \ref{Vector Space Methode} en \ref{Technieken voor Pre-Processing} werd uitgelegd hoe we de data konden meegeven en optimaliseren voor het zelflerende algoritme. Vervolgens werd er in  \ref{Leermethode} ingegaan op de zelflerende algoritmes, waarbij de  Naive Bayes Classifier en Beslissingsbomen werden besproken. We zagen ook bias en variantie, twee begrippen uit de machine learning waar men rekening mee moet houden tijdens de analyse.\\
Vervolgens In hoofdstuk \ref{Experiment} kwamen we tot het experiment. Voor het experiment trachtten we een eigen gevoelsanalyse uit te voeren op Nederlandse tekst aan de hand van eenvoudige machine learning technieken. Concreter gingen we film- ,boek- en muziekrecensies analyseren en bepalen of de recensie positief of negatief is. Voor de analyse werd er beroep gedaan op de Naive Bayes Classifier. Met de Naive Bayes Classifier hebben met alle mogelijke permutaties tussen traingings- en testset ge\"experimenteerd. In \ref{Conclusie experiment} hebben we dan alle resultaten naast elkaar gelegd en vergelekenen. Daaruit hebben we dan kunnen besluiten dat het mogelijk is om een gevoelsanalyse aan de hand van de Naive Bayes Classifier uit te voeren. Waarbij de beste prestatie zich voordoet wanneer de trainingsset en testset over hetzelfde onderwerp gaan. Als laatste hebben we ook opgemerkt dat over het algemeen positieve recensies juister ge\"identificeerd worden.
