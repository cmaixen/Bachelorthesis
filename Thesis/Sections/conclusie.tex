\chapter{Conclusie}\label{Conclusie}

In de experimentele analyse hebben we een zo\’n algemeen beeld proberen te vormen over Nederlandse gevoelsanalyse. We hebben een directe vergelijking gemaakt met Engelse gevoelsanalyse en hieruit konden we besluiten dat de technieken voor Engelse gevoelsanalyse gelijk toepasbaar zijn op Nederlandse gevoelsanalyse. We zagen in beide analyseresultaten dezelfde trends met voor beide talen de Naive Bayes Classifier in combinatie met het verwijderen van stopwoorden , Term weighting en Bigrams als beste techniek. Vervolgens hebben we classificatie onderzocht op basis van geannoteerde woordenlijsten. Hieruit hebben we besloten dat het Engels woordenlijsten niet transparant vertaald kunnen worden naar het Nederlands en de classificatie onvoldoende presteert. We zagen hier deels de oorzaak lag bij een andere woordenschat, leenwoorden, schrijffouten, internetslang en uitgesmeerde woorden. In verder onderzoek kan men deels deze invloeden wegnemen door de woordenlijsten zelf samen te stellen op basis van een Nederlandse dataset en hier de prestatie van te onderzoeken. Als laatste hebben we de invloed van jargon onderzocht bij Nederlandse gevoelsanalyse. Hier zagen we dat wanneer men classifier traint voor een bepaald jargon deze ook het beste presteert voor dat jargon. Al zagen we dat desondanks het jargon een algemeen begrip, in ons geval het onderscheiden van een positieve en negatieve opinie, kan aangenomen worden door de classifier. 