\chapter{Introductie}\label{introductie}
Vandaag de dag is informatie nog nooit zo belangrijk geweest. De afgelopen jaren is er een explosie aan data ontstaan waarbij iedere dag enorm veel data bijkomt. Men durft zelfs spreken over dit huidige tijdperk als \textit{``The Age of Big Data''} \cite{lohr2012age}. Neem social media, waar iedere dag duizende gebruikers hun mening uiten over alledaags dingen. De grootste uitdaging bestaat echter uit de grote hoeveelheid data te kunnen analyseren en daar een zekere kennis uit te vergaren. Vanwege de grote hoeveelheid aan data is het onmogelijk om voor data-analyse manueel een programma te schrijven. Machine learning biedt hier de oplossing. Dit is een onderzoeksdomein binnen de Artifici\"ele Intelligentie dat zich toespitst op zelflerende algoritmes. In deze bachelorproef gaan we bekijken hoe we een data-analyse kunnen uitvoeren op Big data. In hoofdstuk \ref{Lectuur} worden het onderzoeksdomein en de technieken voor data-analyse toegelicht. Na hoofdstuk \ref{Lectuur} volgt er een experiment in hoofdstuk \ref{Experiment}, waar we de data-analyse effectief uitvoeren en bespreken.
\newline
Specifieker is de data-analyse die we gaan uitvoeren op de Big Data gevoelsanalyse, ook wel Sentiment Analysis genoemd, waarbij we een onderscheidt willen maken tussen positieve en negatieve Nederlandse tekst. De keuze voor Nederlandse tekst komt van het feit dat er al enorm veel onderzoek gedaan is naar Sentiment Analysis op Engelse tekst bijvoorbeeld \cite{pang2008opinion} , maar veel minder op Nederlands tekst. Wat het interessant maakt dit eens te onderzoeken. 
\newline
Voor dat we aan het experiment konden beginnen moest er eerst bepaald worden op welke Big Data we de analyse gingen uitvoeren. Zoals eerder vermeld vindt men op sociale media, meer bepaald Twitter, enorm veel informatie en dit was het eerst uitgangspunt voor het verzamelen van data. Maar een groot nadeel van data uit sociale media is sarcasme. Wat soms al moeilijk is voor mensen om sarcasme te detecteren is het zeker voor een algoritme. Wat maakt dat data afkomstig van sociale media niet gunstig is voor de training van het zelflerende algoritme. Als oplossing is er gekozen om data te gebruiken van film- , muziek- en boekrecensies. Meer over waarom we voor deze data hebben gekozen en hoe we deze verzameld hebben vindt men in \ref{De Dataset}.
\newline
Samengevat is dit document opgesplitst in drie delen. Met in hoofdstuk \ref{Lectuur} alle theorie en methodes die van belang zijn voor het experiment. Vervolgens in hoofdstuk\ref{Experiment} het experiment waarbij we aan de hand van data analyse op Nederlandse film- , boek, muziekrecensie trachten positieve en negatieve recensie te bepalen. Ten slotte in hoofdstuk \ref{Conclusie} vormen we een conclusie over het experiment en of het al dan niet mogelijk is om een succesvolle gevoelsanalyse uit te voeren op Nederlandse tekst.