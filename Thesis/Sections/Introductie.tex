\chapter{Introductie}\label{introductie}
Vandaag de dag is informatie nog nooit zo belangrijk geweest. Iedere dag komt er ook enorm veel informatie bij. Kijk maar naar social media, waar iedere dag duizende gebruikers hun mening uiten over alledaags dingen. Het is dan ook zeer interessant om die data te analyseren en daar een zekere kennis uit te vergaren. Vanwege de grote hoeveelheid aan data is het onmogelijk om een programma manueel te schrijven, dat enige kennis uit die data kan halen. Machine learning biedt hier de oplossing. Dit is een onderzoeksdomein binnen de Artifici\"ele Intelligentie dat zich toespitst op zelflerende algoritmes. In deze Bachelorproef onderzoeken we of we met enkele technieken uit de machine learning een goede analyse kunnen uitvoeren op tekst. Concreter wordt er gefocust op de gevoelsanalyse van Nederlandse tekst, waarbij een programma beslist of de gegeven tekst een positief of negatief gevoel uitdrukt.
%
Zoals vermeld voeren we het onderzoek uit met technieken uit de Machine Learning, specifieker binnen de Machine learning situeren we ons onder supervised learning, waarbij we het algoritme trainen met een dataset die voorbeelden bevat over het concept dat we willen aanleren. De trainingsset bevat zowel de inputwaarden als de verwachte outputwaarde voor de input en men verwacht dat het zelflerende algoritme hier verbanden in kan vinden zodanig dat het voor willekeurige inputwaarden de juiste outputwaarde kan bepalen. \\
%
We in dit onderzoek een gevoel af te leiden, wat men kan beschouwen een het labelen van tekst met een bepaald. Wat ons binnen de machine learning tot een classificatieprobleem brengt waarbij verwachte outputwaarde voor bepaalde inputwaarden zich beperkt tot een kleine set van mogelijkheden.\\
%
Nu dat we weten waar we ons precies situeren namelijk binnen de Machine learning en meer bepaald bij supervised learning met classificatieproblemen, kunnen we de vraag stellen welke data we juist nodig hebben voor de gevoelsanalyse op uit te voeren. We hadden al bepaald dat we deze uitvoeren op Nederlandse tekst, maar dit is niet specifiek genoeg. Het vinden van Nederlandse tekst is geen probleem. Men kan een grote hoeveelheid tekst uit krantenartikels, boeken, handleidingen verzamelen, maar dit volstaat niet. Er moet specifiek gezocht worden naar teksten die duidelijke informatie geeft over een gevoel of een emotie en ze moet duidelijk met dat gevoel gelabeld zijn zodanig dat het algoritme hieruit kan leren. De keuze om de gevoelens te veralgemenen en onder te verdelen in twee groepen namelijk positieve en negatieve gevoelens, geeft ons duidelijke labeling. Verder geeft het inzicht over waar men de Nederlandse tekst moet gaan zoeken. Een eerste idee was het verzamelen van tweets rond een bepaalde case bijvoorbeeld de nieuwe uurregeling van de NMBS en deze manueel te labelen of eventueel de labeling te automatiseren aan de hand van hashtags. Na het nader bekijken van deze dataset is gebleken dat er nogal veel sarcasme heerst op Twitter en niet de informatie bevat die we zoeken. Het verzamelen van reviews en meer bepaald over films, boeken en muziek biedt de oplossing. Reviews bieden alles wat men nodig heeft voor het onderzoek. Een review is of wel positief of negatief en de rating die meestal aanwezig is bij een review stelt ons in staat om de review correct te labelen. De keuze om films, boeken en muziek is een beredeneerde keuze. Het aanbod is enorm, meestal niet te specifiek en het is toegankelijk. De datasets die in deze bachelorproef worden gebruikt zijn afkomstig van moviemeter.be, boekmeter.be en muziekmeter.be . Productreviews was ook een mogelijke optie, maar deze bleken te specifiek te zijn waardoor het moeilijk wordt om het algoritme op algemeen sentiment te trainen. \\  
%
Samengevat is deze thesis opgesplitst in drie delen namelijke de achtergrondinformatie, het onderzoek en de conclusie. Als achtergrondinformatie worden alle technieken die tijdens die het onderzoek worden toegepast besproken. Zoals eerder besproken zijn dit technieken die zich onder supervised learning situeren en een oplossing bieden voor classificatie problemen. Vervolgens bespreken we het onderzoek, waarbij we  de reviews van films, boeken en muziek als datatset nemen. Als laatste vormen we een conclusie over de mogelijkheid om met enkele technieken uit de machine learning een geslaagde gevoelsanalyse kunnen uitvoeren op Nederlandse tekst.  
