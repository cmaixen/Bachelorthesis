\chapter{Conclusie}\label{Conclusie}

Voor deze bachelorproef gingen we onderzoeken of de werkwijzen voor Engelstalige gevoelsanalyse ook toepasbaar zijn voor Nederlandstalige gevoelsanalyse en proberen we deze verschillen beter te specificeren. We hebben getracht om aan de hand van een experimentele analyse hier een antwoord op te vinden.
In de experimentele analyse hebben we een algemeen beeld proberen te vormen over Nederlandse gevoelsanalyse. We hebben een directe vergelijking gemaakt met Engelse gevoelsanalyse. In deze vergelijking zagen we dat Engelse gevoelsanalyse in het algemeen beter presteerde dan Nederlandse gevoelsanalyse, maar dat voor beide gevoelsanalyses er goede resultaten werden behaald. En de technieken voor Engelse gevoelsanalyse ook toepasbaar zijn op Nederlandse gevoelsanalyse. We zijn mogelijke oorzaken van die betere prestatie voor het Engels gaan onderzoeken. Hieruit konden we besluiten dat het gemiddelde aantal woorden voor een recensie een invloed hebben op de betere prestatie van Engelse gevoelsanalyse, maar niet de oorzaak is van de betere prestatie.  
Ook zagen we in beide analyseresultaten dezelfde trends. Zo zagen we voor beide talen de Naive Bayes Classifier in combinatie met het verwijderen van stopwoorden, Term weighting en Bigrams als beste techniek.\\

Vervolgens hebben we classificatie onderzocht op basis van geannoteerde woordenlijsten van gevoelens. Hier hadden we voor de Nederlandse woordenlijsten, een vertaling gebruikt van de Engelse woordenlijsten. Hier konden we besluiten dat Engels woordenlijsten niet transparant vertaald kunnen worden naar het Nederlands en de classificatie onvoldoende presteert. We zagen hier deels de oorzaak lag bij een andere woordenschat, leenwoorden, schrijffouten, internetslang en uitgesmeerde woorden. In verder onderzoek kan men deels deze invloeden wegnemen door de woordenlijsten zelf samen te stellen op basis van een Nederlandse dataset en hier de prestatie van te onderzoeken.\\

Als laatste hebben we de invloed van jargon onderzocht bij Nederlandse gevoelsanalyse. Hier zagen we dat wanneer men een classifier traint voor een bepaald jargon deze ook het beste presteert voor dat jargon. Ook zagen we dat een algemeen concept, het onderscheiden van een positieve en negatieve opinie, kan aangenomen worden door de classifier, desondanks het jargon in de datasets.\\

We kunnen dus besluiten dat de technieken voor Engelse gevoelsanalyse ook toepasbaar voor Nederlandse gevoelsanalyse. Verder onderzoek is nog nodig naar de betere prestatie bij Engelse gevoelsanalyse. We hebben in deze bachelorproef enkele pistes doorlopen, maar zonder sluitend antwoord.