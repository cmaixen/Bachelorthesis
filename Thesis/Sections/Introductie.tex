\chapter{Introductie}\label{introductie}
Vandaag de dag is digitale informatie een zeer belangrijk item. We worden als het ware overspoeld door de explosie aan data. En iedere dag wordt deze data groter en groter. Men durft zelfs spreken over dit huidige tijdperk als \textit{``The Age of Big Data''} (\cite{lohr2012age}). Neem social media, waar iedere dag duizenden gebruikers hun mening uiten over alledaagse dingen. De grootste uitdaging bestaat erin om uit die grote hoeveelheid data een analyse te maken en daar de nodige kennis uit te vergaren. Vanwege de grote hoeveelheid aan data is het onmogelijk om voor een data-analyse manueel een programma te schrijven. Machine learning biedt hier de oplossing. Dit is een onderzoeksdomein binnen de Artifici\"ele Intelligentie dat zich toespitst op zelflerende algoritmes. In deze bachelorproef gaan we bekijken hoe we een data-analyse kunnen uitvoeren op een grote dataset. In hoofdstuk \ref{Lectuur} worden het onderzoeksdomein en de technieken voor data-analyse toegelicht. Na hoofdstuk \ref{Lectuur} volgt er een experiment in hoofdstuk \ref{Experiment}, waar we de data-analyse effectief uitvoeren en bespreken.
\newline
Specifieker is de data-analyse die we gaan uitvoeren op de grote dataset gevoelsanalyse, ook wel Sentiment Analysis genoemd, waarbij we een onderscheid willen maken tussen positieve en negatieve Nederlandse tekst. Doordat er al veel onderzoek is verricht naar Sentiment Analysis op Engelse tekst, bijvoorbeeld \cite{pang2008opinion}, is hier in deze bachelorproef de voorkeur gegeven aan een analyse voor Nederlandse tekst.
\newline
Voor dat we aan het experiment konden beginnen moest er eerst bepaald worden op welke grote dataset we de analyse gingen uitvoeren. Zoals eerder vermeld vindt men op sociale media, meer bepaald Twitter, enorm veel informatie en dit was het eerst uitgangspunt voor het verzamelen van data. Maar een groot nadeel van data uit sociale media is sarcasme. Waar het soms al moeilijk is voor mensen om sarcasme te detecteren, is dit het zeker voor een algoritme. Wat maakt dat data afkomstig van sociale media niet gunstig is voor de training van het zelflerende algoritme. Als oplossing is er gekozen om data te gebruiken van film- , muziek- en boekrecensies. In hoofdstuk \ref{De Dataset} vindt men meer over waarom we voor deze data hebben gekozen en hoe we deze verzameld hebben.
\newline
Samengevat is dit document opgesplitst in drie delen. Met in hoofdstuk \ref{Lectuur} alle theorie en methodes die van belang zijn voor het experiment. Vervolgens in hoofdstuk\ref{Experiment} het experiment waarbij we aan de hand van data analyse op Nederlandse film- , boek, muziekrecensie trachten positieve en negatieve recensie te bepalen. Ten slotte in hoofdstuk \ref{Conclusie} vormen we een conclusie over het experiment en of het al dan niet mogelijk is om een succesvolle gevoelsanalyse uit te voeren op Nederlandse tekst.