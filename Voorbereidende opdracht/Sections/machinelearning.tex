\section{Machine Learning}\label{Machine Learning}

Machine learning is een welgekend begrip in de informatica wereld, maar wat het juist omvat, zijn toepassingen en hoe het helpt om de juiste verbanden te achterhalen uit een enorme datasets wordt uitgelegd in dit hoofdstuk.

\subsection{Wat is Machine Learning}\label{Wat is Machine Learning}

Over wat Machine Learnig juist is, vindt men nergens een eenduidige definitie. Vele hebben hebben geprobeerd om een eenduidige definitie te definieren, zoals Arthur Samuel(1959). Hij definieerde Machine Learning als \begin{quote}Field of study that gives computers the ability to learn without being explicitly programmed\end{quote}. Later heeft Tom Michel(1999) ook een poging ondernomen en stelde een well-posed learning problem als het volgende \begin{quote} A computer program is said to learn from experience E with respect to some class of tasks T and performance measure P, if its performance at tasks in T, as measured by P, improves with experience E. \end{quote}. Als we al deze pogingen proberen te omvatten, kunnen we Machine Learning het best beschrijven als een onderzoeksdomein dat zich bezighoudt met het onderzoeken en de ontwikkeling van zelflerende algorithmes.
\\
Binnen Machine Learning kan men verschillende groepen van lerende algorithmes onderscheiden. Zo heeft men Supervised learning, Unsupervised learning, Reinforcement learning en Recommender systems. In deze voorbereiding gaat men zich enkel opleggen op supervised en unsupervised learning. Deze soorten algorithmes gaat men de nodige antwoorden bezorgen om later gevoelsanalyses te kunnen uitvoeren.


\subsection{Supervised Learning}\label{Supervised Learning}
\subsection{Unsupervised Learning}\label{Unsupervised Learning}


\subsection{Data Mining}\label{Data Mining}

