\chapter{Conclusie}\label{Conclusie}

Voor deze bachelorproef gingen we onderzoeken of de werkwijzen voor engelstalige gevoelsanalyse ook toepasbaar zijn voor nederlandstalige gevoelsanalyse en proberen we deze verschillen beter te specificeren. We hebben getracht om aan de hand van een experimentele analyse hier een antwoord op te vinden.
In de experimentele analyse hebben we een algemeen beeld proberen te vormen over Nederlandse gevoelsanalyse. We hebben in \ref{Engelse gevoelsanalyse versus Nederlandse Gevoelsanalyse} een directe vergelijking gemaakt met Engelse gevoelsanalyse. In deze vergelijking zagen we dat Engelse gevoelsanalyse in het algemeen beter presteerde dan Nederlandse gevoelsanalyse, maar dat voor beide gevoelsanalyses er goede resultaten werden behaald. En de technieken voor Engelse gevoelsanalyse wel degelijk overdraagbaar zijn naar Nederlandse gevoelsanalyse. We zijn mogelijke oorzaken van die betere prestatie voor het Engels gaan onderzoeken. Hieruit konden we besluiten dat de impact van de hoeveelheid woorden (data) in een tekst een rol spelen in de prestatie. We zagen een duidelijk prestatie voordeel bij de Engelse dataset met langere woorden. Dit bevestigd nogmaals dat een goed presterende gevoelsanalyse met de huidige technieken enkel mogelijk is voor meer substanti\"ele teksten, en moeilijk tot onmogelijk voor kortere stukken tekst.\\
Ook zagen we in beide analyseresultaten dezelfde trends. Zo zagen we voor beide talen de Naive Bayes Classifier in combinatie met het verwijderen van stopwoorden, Term weighting en Bigrams als beste techniek en zagen we dezelfde prestatieverschillen tussen de algoritmen mee overgaan van het Engels naar het Nederlands.\\

Vervolgens hebben we classificatie onderzocht op basis van geannoteerde woordenlijsten van gevoelens. Hier hadden we voor de Nederlandse woordenlijsten, een vertaling gebruikt van de Engelse woordenlijsten. Hier konden we besluiten dat Engels woordenlijsten niet transparant vertaald kunnen worden naar het Nederlands en de classificatie onvoldoende presteert. We zagen dat hier deels de oorzaak lag bij een andere woordenschat, leenwoorden, schrijffouten, internetslang en uitgesmeerde woorden. In verder onderzoek kan men deels deze invloeden wegnemen door de woordenlijsten zelf samen te stellen op basis van een Nederlandse dataset en hier de prestatie van te onderzoeken. Een andere opvallende bevinding uit dit experiment is de opvallende overeenkomst van negatieve engelstalige woorden in onze Nederlandse dataset. Dit kan ook een gevolg zijn van de herkomst van onze Nederlandse dataset (een ``internetpubliek'' dat onder andere veel gebruik maakt van anglicismen).\\

Als laatste hebben we de invloed van jargon onderzocht bij Nederlandse gevoelsanalyse. Hier zagen we dat wanneer men een classifier traint voor een bepaald jargon deze ook het beste presteert voor dat jargon. Ook zagen we dat een algemeen concept, het onderscheiden van een positieve en negatieve opinie, kan aangenomen worden door de classifier, desondanks het jargon in de datasets.\\

We hebben aangetoond dat technieken voor gevoelsanalyse grotendeels overdraagbaar zijn, al is het belangrijk om rekening te houden met het feit dat technieken die afhangen van uitgebreid geannoteerde woordenlijsten vaak niet rechtstreeks te vertalen zijn. Voor deze aanpakken is het dus nodig om per taal aangepaste woordenlijsten op te stellen. Maar gezien de goede prestaties van technieken die zonder deze lijsten kunnen werken, kunnen we stellen dat het vaak interessanter is om te investeren in de ontwikkeling van een goede herbruikbare leertechniek in plaats van een uitgebreide geannoteerde woordenlijst per taal.