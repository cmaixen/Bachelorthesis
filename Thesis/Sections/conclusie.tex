\chapter{Conclusie}\label{Conclusie}

In deze bachelorproef hadden we als onderwerp gevoelsanalyse in het Nederlands. De theorie gaf ons de basis voor het onderzoek, waarbij werd uitgelegd hoe we de gevoelsanalyse konden uitvoeren aan de hand van machine learning. Er werd toegelicht hoe we de data konden meegeven en optimaliseren voor het zelflerende algoritme. Vervolgens werd er ingegaan op de zelflerende algoritmes, waarbij de  Naive Bayes Classifier en Beslissingsbomen werden besproken. We zagen ook bias en variantie, twee begrippen uit de machine learning waar men rekening mee moet houden tijdens de analyse.\\
Vervolgens kwamen we tot het experiment in hoofdstuk \ref{Experiment}. Voor het experiment trachtten we een eigen gevoelsanalyse uit te voeren op Nederlandse tekst aan de hand van eenvoudige machine learning technieken. Concreter gingen we film- ,boek- en muziekrecensies analyseren en bepalen of de recensie positief of negatief is. Voor de analyse werd er beroep gedaan op de Naive Bayes Classifier. Met de Naive Bayes Classifier hebben met alle mogelijke permutaties tussen trainings- en testset ge\"experimenteerd. Als we alle resultaten naast elkaar leggen en vergelijken kunnen we besluiten dat het mogelijk is om een gevoelsanalyse aan de hand van de Naive Bayes Classifier uit te voeren, waarbij de beste prestatie zich voordoet wanneer de trainingsset en testset over hetzelfde onderwerp gaan. Bijvoorbeeld wanneer men het algoritme traint op muziekrecensies en vervolgens een gevoelsanalyse uitvoert op muziekrecensies. Als laatste hebben we ook opgemerkt dat over het algemeen positieve recensies juister ge\"identificeerd worden.
