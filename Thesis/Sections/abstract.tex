\begin{abstract}
Gevoelsanalyse is een populaire gegeven binnen de Machine Learning. In deze bachelorproef gaan we op zoek of het mogelijk is om aan de hand van enkele eenvoudige machine learning technieken een gevoelsanalyse uit te voeren. Specifieker focussen we op de Nederlandse taal, waar het naslagwerk vandaag de dag eerder beperkt van is. Als onderwerp van de gevoelsanalyse worden film-,boek- en muziekrecensies aan de hand van een algoritme beoordeeld of ze een positieve of negatieve emotie uitdrukken. Voor het onderzoek bekijken we de theoretische kant van een gevoelsanalyse, waar we de mogelijke technieken bespreken. Daarnaast wordt ook de praktische zijde uitgewerkt waar we de theoretische kennis gaan omzetten in een experiment. Dit experiment toont aan dat het mogelijk is om gevoelsanalyse uit te voeren op het Nederlands.
\end{abstract}
