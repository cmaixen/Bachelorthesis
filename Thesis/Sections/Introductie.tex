\chapter{Introductie}\label{introductie}

Vandaag de dag beschikken we over een enorme hoeveelheid aan digitale informatie. Ook wordt deze hoeveelheid aan informatie iedere dag groter en groter. 
In deze \textit{``Age of Big Data''} (\cite{lohr2012age}) bestaat de uitdaging erin om uit deze grote hoeveelheid data, door middel van analyse bepaalde inzichten te krijgen.
\\
Vele hebben dit probleem proberen aan te pakken, waarbij men zich vooral bezig hield met het brengen van structuur in deze grote dataset en zich voornamelijk concentreerde op onderwerp-gebaseerde classificatie. Echter met de opkomst van sociale media, blogs, reviewsites is er een groeiende interesse ontstaan voor gevoelsanalyse. Het onderzoek dat hiernaar gebeurd, heeft ook de nodige aandacht van bedrijven en ook zij spelen hier een rol in.\\
%substukken
Als we over gevoelsanalyse spreken dan refereren we naar het verwerken van natuurlijke taal  om zo via tekstanalyse en Computationele taalkunde subjectieve informatie uit te tekst te kunnen halen. Volgend voorbeeld illustreert een tweet waarop men bijvoorbeeld gevoelsanalyse kan uitvoeren.\\

\begin{figure}[h]%
    \centering
    \subfloat{{\includegraphics[width=10cm]{slechtedag} }}%
    \caption{een voorbeeldtweet voor gevoelsanalyse}%
\end{figure}
\newline

Verschillende technieken zijn hier mogelijk  als gevoelsanalyse om de subjectiviteit of opinie uit deze tweet te bepalen. Men kan zich baseren op het woord \'slecht' en zo vast stellen dat de tweet een negatieve emotie voorstelt. Maar men kan zich ook baseren op eerder vastgestelde tweets en op basis hiervan een beslissing nemen. Echter stel dat de gegeven persoon juist de lotto had gewonnen en zich sarcastisch uitdrukte. Dit zijn problemen waar men vandaag de dag nog altijd niet uit is met gevoelsanalyse.\\ 

Nu bijna al het onderzoek dat de afgelopen jaren gebeurd is met betrekking tot gevoelsanalyse werd uitgevoerd op Engelse teksten en daarom is er zeer weinig te vinden over onderzoek met betrekking tot gevoelsanalyse op het Nederlands. Deels omdat het Engels een wereldtaal is en het Nederlands niet, maar ook doordat de bedrijven die dergelijk onderzoek uitvoeren op het Nederlands hun onderzoek (resultaten) binnenshuis houden.
\\
Voor deze bachelorproef kijken we of de bevindingen over gevoelsanalyse op het Engels gelijk toepasbaar zijn op het Nederlands en/of er verschillen zijn tussen het Nederlands en het Engels, waarbij men met een gevoelsanalyse rekening mee dient te houden.\\
%workflow van wat je hebt gedaan
Als eerste hebben we samen met de voorbereiding van de bachelorproef in het 1ste semester ons verder verdiept in de literatuur over Engelse gevoelsanalyse en maakte ons vertrouwd met de programmeertaal python, een van de programmeertalen bij uitstek die ons toelaat om experimentele analyses op te zetten. Vervolgens zijn we op zoek gegaan naar onze data. Voor het de gevoelsanalyse te kunnen vergelijken, moet men zowel over een Engelse dataset als een Nederlandse dataset beschikken. Door het vele onderzoek naar Engelse gevoelsanalyse zijn er voldoende dataset beschikbaar op het web. Echter Nederlandse datasets, gelabeld volgens gevoel, zijn heel moeilijk te vinden en dwingt ons om de data manueel te scrapen. Voor het scrapen moeten we zoals eerder vermelde opletten voor sarcasme. Dit sluit scrapingbronnen zoals Twitter en andere sociale media volledig uit. Gelukkig bieden reviewsites als \url{www.moviemeter.be} de oplossing en gebruiken we deze sites als bron voor te scrapen. In sectie \ref{De Dataset} gaan we verder in op de scraping en verzamelde dataset. Na dat we de nodige kennis, vaardigheden en datasets hebben, bepalen we de technieken die we gebruiken voor de gevoelsanalyse en de vergelijking in onze experimentele analyse. Initi\"eel om goed te doorgronden wat er juist gebeurd tijdens deze technieken, implementeren we deze technieken zelf voor de experimentele analyse. Later bij het dooranalyseren, optimaliseren we de code door gebruik te maken van de bibliotheek sklearn (\url{http://scikit-learn.org/}), die ons toelaat om sneller een experimentele analyse uit te voeren.\\   
%overzicht van wat men voorgeschoteld krijgt
%
Samengevat bespreken we in hoofdstuk \ref{Lectuur} de technieken die we gebruiken voor de experimentele analyse met hun theoretische achtergrond. \\Vervolgens gaan we in hoofdstuk \ref{Experiment} over naar de experimentele analyse.\\ In deze analyse vergelijken de resultaten van de technieken op Nederlandse en Engelse filmrecensies.\\
Vervolgens concentreren we ons op de woordenschat en  onderzoeken we de classificatie op basis van een Engels en Nederlands geannoteerde woordenlijst van gevoelens.\\
Afhankelijk van het positieve karakter van de resultaten gaan we nog iets dieper in Nederlandse gevoelsanalyse.\\
Na het experimentele analyse vormen we een conclusie in hoofdstuk \ref{Conclusie} of de gevonden technieken voor gevoelsanalyse op het Engels al dan niet gelijk toepasbaar zijn op het Nederlands. 