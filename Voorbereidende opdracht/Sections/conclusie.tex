\chapter{Conclusie}\label{Conclusie}

Nu men weet wat machinelearning juist omvat, welke technieken er zijn en hoe men specifiek text mining moet aanpakken,kan men dit koppelen aan de bachelorproef. De bachelorproef omvat het onderzoeken en toepassen van een gevoelsanalyse op sociale media. Sociale media is in wezen \'e\'en grote set aan data, meerbepaald een set met tekst data. Het doel van gevoelsanalyse is verbanden ontdekken in de dataset om zo te kunnen classifiseren op gevoelens. Dit wijst erop dat men text mining kan gebruiken om zo informatie te verkrijgen over de dataset. Het bevat alle middelen om de gevoelsanalyse op sociale media toe te passen. Ten eerst moet men uiteraard de data verzamelen. Aangezien men te maken heeft met een dataset waarvan men geen informatie heeft, moet men technieken gebruiken van unsupervised learning. Men gaat alle optimalisaties gebruiken om zo een optimaal resultaat te krijgen. Dus men gaat de data pre-processen en vervolgens een verfijnd vector space model analyseren. Het vector space model kan verfijnd worden door de geziene technieken als latent semantic analysis en term weighting. Ten slotte moet men de resultaten verwerken met een cluster algorithme. De sequentie van verzamelen, verwerken en analyseren is een zeer belangrijk gegeven bij onze gevoelsanalyse.


%nog niet helemaal tevreden, heb evnetueel wat input nodig.