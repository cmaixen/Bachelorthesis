\chapter{Introductie}\label{introductie}
Vandaag de dag is informatie nog nooit zo belangrijk geweest. Iedere dag komt er ook enorm veel informatie bij. Kijk maar naar social media, waar iedere dag duizende gebruikers hun mening uiten over alledaags dingen. Het is dan ook zeer interessant om die data te analyseren en daar een zekere kennis uit te vergaren. Vanwege de grote hoeveelheid aan data is het onmogelijk om een programma manueel te schrijven, dat enige kennis uit die data kan halen. Machine learning biedt hier de oplossing. Dit is een onderzoeksdomein binnen de Artifici\"ele Intelligentie dat zich toespitst op zelflerende algoritmes. Deze voorbereiding bespreekt de technieken binnen de machine learning die ons kunnen helpen voor data- en meer bepaald text mining. Er volgt eerst een algemene introductie over machine learning en de technieken. Vervolgens bespreken we specifiekere technieken, met de focus op text mining en als laatste koppelen we de technieken aan de eigelijke bachelorproef namelijk gevoelsanalyse op sociale media.
