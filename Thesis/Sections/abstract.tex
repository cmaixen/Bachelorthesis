\begin{abstract}

Gevoelsanalyse is een populaire gegeven binnen de Machine Learning. Over gevoelsanalyse op de Engelse taal vindt men voldoende naslag werk (http://nlp.stanford.edu/sentiment/), maar in het Nederlands is dit eerder beperkt. Dit heeft deels te maken met het kleine bereik van de Nederlandse taal wat het onderzoek ernaar al minder interessant maakt en als men er onderzoek naar zou doen, dit vaak door een bedrijf wordt uitgevoerd waarbij de resultaten onder bedrijfsgeheim vallen. Daarom concentreren wee bij deze bachelorproef op gevoelsanalyse op de Nederlandse taal. We experimenteren met enkele algemene technieken uit de Machine Learning die ons een degelijke gevoelsanalyse kunnen opleveren en proberen ten slotte een besluit te trekken of we met enkele algemene technieken effectief een acceptabele gevoelsanalyse op het Nederlands kunnen uitvoeren.


\end{abstract}
