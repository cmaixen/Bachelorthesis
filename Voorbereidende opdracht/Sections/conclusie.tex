\chapter{Conclusie}\label{Conclusie}

In deze voorbereiding hebben we omschreven wat machinelearning juist omvat, namelijk het onderzoeken en ontwikkeling van zelflerende algoritmes. Waarbij het hoofdzakelijk uit drie stappen bestaat, namelijk data verzamelen, verwerken en analyseren. Naargelang het soort data en wat deze weergeeft bestaan er binnen het domein van Machine Learning verschillende technieken met hun specifieke eigenschappen en voorbeelden.  Voor situaties waarin we op voorhand over voldoende data beschikken en we duidelijk weten wat deze data betekend, bespraken we in sectie 2.2 verschillende supervised learning technieken die in staat zijn om een hypothese te formuleren op basis van de gegeven data. We maakte een duidelijk onderscheid welke problemen er zich kunnen voordoen zoals een classificatie probleem versus een regressie probleem en hoe men deze moet oplossen.\\
Verder hebben we een specifieke techniek besproken, namelijk de vector space methode die van toepassing is bij text mining en hoe men specifiek text mining moet aanpakken. Zo is het aangeraden  in sectie 3.1 om document pre-processing toe te passen op de dataset voor het te laten verwerken door het algoritme en in sectie 3.2.1 de standaard vector space methode te optimaliseren met technieken zoals Latent Semantic Analysis. Ten slotte hebben een proefopstelling opgesteld waarbij we de techniek LSA toepassen en  waarbij de efficientie en werking van Latent Semantic Analysis nogmaals wordt bevestigd.

