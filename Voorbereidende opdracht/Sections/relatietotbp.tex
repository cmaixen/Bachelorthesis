\chapter{Beschrijving Bachelorproef}\label{Beschrijving Bachelorproef}

In de bachelorproef gaan we meerdere technieken voor gevoelsanalyse trainen en evalueren specifiek voor de Nederlandse taal. Deze technieken gaan van het eenvoudig gebruik maken van bestaande woordlijsten tot het trainen van een classificatiealgortime met nieuwe verzamelde data zoals bijvoorbeeld resensies van Nederlandse magazines en websites. De prestatie van deze technieken wordt geevalueerd door het te testen op dataset zoals het NMBS experiment waarbij men gaat onderzoeken of bepaalde waarden van de gevoelsanalyse verschillende twittergebruikters kunnen onderscheiden bijvoorbeeld diegene die klagen versus diegene die informeren. Meer concreet bestaat de proef uit meerdere taken. Eerst verzamelt men trainingsdata van Twitter en labelt men manueel de data voor een classifier. Door de trainingsset te labelen, hebben we te maken met supervised learning. Uit de voorbereiding weten we welke problemen er zich kunnen voordoen, hoe deze op te lossen. Vervolgens moet men trainigsdata van verschillende Nederlandse websites verzamelen voor een meer geadvanceeerde training. Bij die geavanceerde trainingen, weten we dankzij ons experiment dat we zeker beroep moeten doen op technieken van text mining zoals Latent Semantic Analysis of term weighting. Indien we dit niet doen, zullen de prestaties veel minder zijn. Om te kijken of de prestaties wel voldoen moet men daaropeenvolgend de prestaties vergelijken met algemene (Engels geoptimaliseerde) technieken. Als laatste test men de invloed van de gevoelsanalyse bij de andere projecten. 
    
