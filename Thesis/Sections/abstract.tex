\begin{abstract}
In de huidige ``Age of Big Data'' (Lohr, 2012) bestaat de grote uitdaging erin om bepaalde inzichten te krijgen door middel van analyse. Via tekstanalyse en computationele taalkunde of  gevoelsanalyse, kan er subjectieve informatie uit een tekst gehaald worden. Bijna al het onderzoek dat tot op heden met betrekking tot gevoelsanalyse is uitgevoerd op Engelse teksten. Voor deze bachelorproef kijken we of de technieken op het Engels ook van toepassing zijn op de Nederlandse taal en wat de verschillen zijn tussen deze twee talen. 
Allereerst hebben we een literatuurstudie uitgevoerd op de Engelse gevoelsanalyse . Met behulp van de programmeertaal Python werden de experimentele analyses opgezet.  De verzamelde data bestaat uit zowel een Engelstalige als een Nederlandstalige dataset die volgens de vector space methode werden voorgesteld. Voor de voorwerking van de  data werden verschillende technieken, baserend op de literatuurstudie, toegepast.  Voor de experimentele analyse werd er gebruikt gemaakt van zelflerende algoritmes uit de Machine learning namelijk de Naive Bayes Classifier en  de Decision Tree. 
De analyses tonen aan dat technieken voor de Engelse gevoelsanalyse ook toepasbaar zijn voor Nederlandse gevoelsanalyse. Verschillende sub-analyses tonen aan dat er verschillen zijn eigen aan de taal.
\end{abstract}
