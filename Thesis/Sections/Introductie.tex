\chapter{Introductie}\label{introductie}
Vandaag de dag is informatie nog nooit zo belangrijk geweest. Iedere dag komt er ook enorm veel informatie bij. Kijk maar naar social media, waar iedere dag duizende gebruikers hun mening uiten over alledaags dingen. Het is dan ook zeer interessant om die data te analyseren en daar een zekere kennis uit te vergaren. Vanwege de grote hoeveelheid aan data is het onmogelijk om een programma manueel te schrijven, dat enige kennis uit die data kan halen. Machine learning biedt hier de oplossing. Dit is een onderzoeksdomein binnen de Artifici\"ele Intelligentie dat zich toespitst op zelflerende algoritmes. In deze Bachelorproef onderzoeken we of we met enkele technieken uit de machine learning een goede analyse kunnen uitvoeren op tekst. Concreter wordt er gefocust op de gevoelsanalyse van Nederlandse tekst, waarbij een programma beslist of de gegeven tekst een positief of negatief gevoel uitdrukt.
Deze thesis is opgesplitst in drie grote delen namelijke de achtergrondinformatie, het experiment en de conclusie. Als achtergrondinformatie worden alle technieken die tijdens die het experiment worden toegepast besproken. Vervolgens bespreken we het experiment en de resultaten. Als laatste vormen we een conclusie over de mogelijkheid om met enkele technieken uit de machine learning een geslaagde gevoelsanalyse kunnen uitvoeren op Nederlandse tekst.  
